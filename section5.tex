\section{関連研究}

情報ダッシュボードのデザイン\cite{few}に関する研究は多くないが、
表示するべき情報を選択する手法\cite{Jones:2015:ECI:2800835.2800963}や、
セルの自動配置手法\cite{Hertzog:2015:BSP:2678025.2701383}などの
研究が存在する。
わかるらんどのダッシュボードは
単純な形状のセルを指定どおりに並べているだけであるが、
よりわかりやすい表示のための配置手法の検討は意義があると思われる。

会議での議論を促進するために
「On Air Forum」\cite{nishida2011}、
「Lock-on-Chat」\cite{nishida2006}
など様々なチャットシステムが提案されているが、
このようなチャットシステムのほとんどは
タイムライン型式で表示が行なわれるようになっており、
情報ダッシュボードのような型式で感情や意見など書き込んで
一覧できるチャットシステムは存在しない。

近年、
消極的な人間でも会議の議論などに参加しやすくするための研究が
消極性研究会 (SIGSHY: Special Interest Group on Shyness and Hesitation around You)
というグループなどを中心に盛んになってきているが\cite{kurihara2016}\cite{nishida2011}、
わかるらんどもこのような方向性の支援システムのひとつだといえるだろう。

% 関連する研究には,
% \begin{itemize}
% \item 情報ダッシュボード
% \item 多くの人がいる場での計算機を用いたコミュニケーション
% \end{itemize}
% に関するものがある
% 
% \subsection{情報ダッシュボードに関する研究}
% 研究としては,2つに区切られたレイアウトのダッシュボードのセルの配置を支援するもの
% \cite{Hertzog:2015:BSP:2678025.2701383}や,
% ある課題の解決のためにどのような情報をダッシュボードに表示するべきか
% \cite{Jones:2015:ECI:2800835.2800963}などが議論されている.
% ダッシュボードに人間の感情や現在の状況を表示するといった研究は今までに行われていない.
% 
% \subsection{多くの人がいる場での計算機を用いたコミュニケーションの研究}
% 「Lock-on-Chat\cite{nishida2006}」は複数の話題に分散した会話を促進するチャットシステムである.
% 先に述べた「On Air Forum」はリアルタイムコンテンツを視聴中のコミュニケーションシステムである.
% 
% また,消極性研究会(SIGSHY: Special Interest Group on Shyness and Hesitation around You)という
% グループが,消極的な人と積極的な人が混在する場のデザインとICT支援について研究しており,
% 大勢の人が集まる場で消極的な人でも誰かと交流することを支援するようなシステム\cite{nishida2011}が
% 研究されていたり,消極性研究に関する書籍\cite{kurihara2016}も出版されている.
