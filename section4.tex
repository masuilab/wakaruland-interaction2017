\section{議論}

\subsection{利用経験}
我々の研究室では『わかるらんど』を約6ヶ月の間利用してきた。

\subsection{チャットシステムとしての利用}

講義やコンファレンスではタイムライン表示のテキストチャットが利用されることが多いが,
以下のような問題があると考えている.

\begin{itemize}
\item 同時に多数が投稿するとすぐに流れていってしまう
\item 投稿数の多い人ばかりが目立ってしまう
\item 投稿しない人は全く投稿しない
\end{itemize}

チャットに限らず会議やコンファレンスなどでも特定の人だけが沢山発言し,発言しない人は全く発言しない状況はよくあることだ.
何かしらリアクションや発言をしたいが,気の利いたことを言わなければならない,的外れなことを言えないという環境が積極的な発言を妨げになっていると考える.
WISS2009のコミュニケーション支援システム「On Air Forum」の実証実験\cite{nishida2011}では1回以上発言した人が約半数であった.


思いつきを発言できる環境を作り1人でも多く参加する人を増やすことが,「多くの人の感情をひと目で把握したい」というわかるらんどの目的の達成に必要である.

わかるらんどは全員の最新の投稿のみを表示するインタフェースである,
スタンプしか投稿できず高度な意見を述べることは全く期待されないことから,

\begin{itemize}
\item 短時間に多くの人が投稿しても流れて見えなくなってしまうことがない
\item 投稿数が多いからといって目立つわけではない
\item 投稿のハードルが低い
\end{itemize}
というテキストチャットやタイムラインにはない特徴がある.

わかるらんどのインタフェースは長い文章を投稿するのに適していないのでわかるらんど上で議論を行うことは難しい.
発表の場合は最後に質問や議論の時間があるので議論はその時に行えばよい.