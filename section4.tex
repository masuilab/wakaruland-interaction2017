\section{議論}

\subsection{チャットシステムとしての利用}

LINE株式会社が提供するスマートフォン向けのコミュニケーションアプリである「LINE」は,日本で非常に多くの人に使われている.
LINEの基本機能はユーザが個人またはグループに対してテキストベースでメッセージを送信できるものであり,
従来のメッセンジャーアプリと機能の面で大きな違いがあるわけではない.
LINEの最大の特徴は「スタンプ」という大型の絵文字のようなピクトグラムを送信できることである.
スタンプはテキストで記述するのが難しい表現や感情を伝えるのに非常に適している.
また,一覧からスタンプを選ぶだけで送信ができるため,テキストを考えて入力するよりも速く簡単である.
近年ではメッセンジャーアプリだけではなくリアルタイムコミュニケーションが必要なオンラインゲームなどでもスタンプの利用が広まっている.

講義やコンファレンスではテキストチャットが利用されることが多いが,
発言にスタンプのみを用いるわかるらんどはチャットシステムが抱える問題を解決できると考える.
チャットシステムには3つの問題があると考える.

\begin{itemize}
\item 同時に多数が投稿するとすぐに流れていってしまう
\item 投稿数の多い人が目立ってしまう
\item 投稿しない人は全く投稿しない
\end{itemize}

チャットに限らず会議やコンファレンスなどでも特定の人だけが沢山発言し,発言しない人は全く発言しない状況はよくあることだ.
何かしらリアクションや発言をしたいが,気の利いたことを言わなければならない,的外れなことを言えないという環境が積極的な発言を妨げになっていると考える.
WISS2009のコミュニケーション支援システム「On Air Forum」の実証実験\cite{nishida2011}では1回以上発言した人が約半数であった.
思いつきを発言できる環境を作り1人でも多く参加する人を増やすことが,「多くの人の感情をひと目で把握したい」というわかるらんどの目的の達成に必要である.

わかるらんどは全員の最新の投稿のみを表示するインタフェースである,
スタンプしか投稿できず高度な意見を述べることは全く期待されないことから,

\begin{itemize}
\item 短時間に多くの人が投稿しても流れて見えなくなってしまうことがない
\item 投稿数が多いからといって目立つわけではない
\item 投稿のハードルが低い
\end{itemize}
というテキストチャットやタイムラインにはない特徴がある.

わかるらんどのインタフェースは長い文章を投稿するのに適していないのでわかるらんど上で議論を行うことは難しい.
発表の場合は最後に質問や議論の時間があるので議論はその時に行えばよい.
そもそも人が発表をしているときはチャットで議論なんてしてないで話を聞くべきである.

\subsection{実際の行動に基づく投稿}

別の作業を行っていてわかるらんどへの投稿をしたいときにブラウザを開いてスタンプを選んで押さなければならない.
前述のボタンやテンキーなど専用の入力装置も作ることができるが,投稿できるものが限られている.
自分が心のなかで「なるほど」と思ったらわかるらんどに「なるほど」と投稿したり,怒ったら怒っている絵文字を投稿したりしたい.
人間の実際の行動に基づいて,膝を打ったら「なるほど」,首を捻ったら「わからん」など慣用句と結びつけた投稿や,
髪をいじったら「考え中」など人の癖を利用した感情の推定も利用できるだろう.