\documentclass[submit,techrep]{ipsj}

\usepackage[dvipdfmx]{graphicx}
\usepackage{latexsym}
\usepackage{url}
\usepackage{here}
\usepackage{listings,jlisting}

\lstset{%
  language={C},
  basicstyle={\small},%
  identifierstyle={\small},%
  commentstyle={\small\itshape},%
  keywordstyle={\small\bfseries},%
  ndkeywordstyle={\small},%
  stringstyle={\small\ttfamily},
  frame={tb},
  breaklines=true,
  columns=[l]{fullflexible},%
  numbers=left,%
  xrightmargin=0zw,%
  xleftmargin=3zw,%
  numberstyle={\scriptsize},%
  stepnumber=1,
  numbersep=1zw,%
  lineskip=-0.5ex%
}

\def\Underline{\setbox0\hbox\bgroup\let\\\endUnderline}
\def\endUnderline{\vphantom{y}\egroup\smash{\underline{\box0}}\\}
\def\|{\verb|}

\setcounter{巻数}{53}%vol53=2012
\setcounter{号数}{10}
\setcounter{page}{1}

% インタラクション特有の設定。印刷工程で柱・ノンブルの埋め込みを行う。
\makeatletter
\pagestyle{empty}
\def\@oddhead{}%
\def\@evenhead{}%
\def\ps@IPSJTITLEheadings{}
\makeatother

\long\def\comment#1{}

\begin{document}

\title{わかるらんど: IoT時代の共有情報視覚化}
\etitle{WakaruLand: An information visualization system for the IoT age}

\affiliate{MG}{慶應義塾大学大学院 政策・メディア研究科}
\affiliate{EI}{慶應義塾大学 環境情報学部}

\author{山田 尚昭}{Naoaki Yamada}{MG}
\author{増井 俊之}{Toshiyuki Masui}{EI}

\begin{abstract}
  全世界のセンサ情報やユーザの気分などを一覧表示したり投稿したりできるシステム「わかるらんど」を提案する。
  %
  ニュースや天気予報のようなリアルタイム情報を並べて一覧する
  「情報ダッシュボード」の利用が広まっているが、
  利用できる情報の種類は限られており、
  ユーザが情報を投稿して共有することはできない。
  %
  わかるらんどは、
  単純で強力なWeb上の情報共有システム「WebLinda」上に構築された
  汎用的な情報共有/視覚化システムであり、
  ユーザの気分を表明したり、チャット文字列を投稿したり、
  センサ情報やWeb上の情報を表示したり、
  ネット上のあらゆる情報を投稿/共有して一覧表示することができる。
  わかるらんどの利用により、
  情報ダッシュボードとSNSやチャットシステムを
  簡単に統合的に利用することができる。
  %
  本論文では、
  わかるらんどの思想及び利用経験について述べ、
  応用について幅広く考察する。
  
  % インフォメーションダッシュボードとスタンプベースのコミュニケーションを組み合わせた視覚化システム『わかるらんど』を提案する。
  % ニュースや天気などの情報はインターネットに流れていて誰でも簡単に見ることができるが、
  % 誰が今何を考えているのか、どこにいて何をしているのかといった情報も知りたい場合、
  % ありとあらゆる情報を簡単にアウトプットできて、それらを見ることができるシステムが必要である。
  % 『わかるらんど』はインフォメーションダッシュボードのセルにスタンプをタイル状に並べて表示する
  % ことで、人の感情や現在の状況、IoT機器の情報などをリアルタイムに視覚化するシステムで、
  % 非常に汎用なダッシュボードとして利用することができる。
  % 『わかるらんど』の実装には、Webサーバ上に実装したLindaシステムである「linda-server」を
  % 使用しており、HTTPが使用できれば様々な環境で利用することができる。
\end{abstract}


\maketitle

\section{はじめに}

% \begin{itemize}
%   \item わかるらんどの思想を述べる
%   \item 消去性コンピューティングについて\cite{kurihara2016}
% \end{itemize}

ワイヤレスネットワークや小型計算機の普及による
IoT社会が到来しつある現在、
人々は大量のリアルタイム情報や通知やメッセージなどに圧倒されている。
%
多くの情報を人間が理解しやすくするため、
以下のような視覚化手法が利用されている。

\vspace{3mm}

\begin{figure}[b]
\centering\fbox{\includegraphics[width=7cm]{images/azure.png}}
\caption{Microsoft Azureの情報ダッシュボード}
\label{azure}
\end{figure}

\paragraph*{情報ダッシュボード}

情報ダッシュボード\cite{few}は、
複数のリアルタイム情報をタイル状に並べて表示することによって
多くの情報をわかりやすく視覚化するシステムである。
たとえばWindowsのスタート画面(図\ref{azure})の情報ダッシュボードには
天気予報や株価のようなリアルタイム情報を表示可能である。

%インフォメーションダッシュボード(図\ref{azure})は、
%単一の画面に複数のリアルタイム情報をタイル状に並べて表示するものだ。
%インフォメーションダッシュボードはセンサの値や株価など常に値が変化していくものをたくさん並べて、
%ひと目で把握するのに非常に便利なインタフェースである。
%しかし、表示領域が限られているため、長いテキストを表示するには適していない。

% ひと目で把握するのに非常に便利なインタフェースである。
% しかし、表示領域が限られているため、長いテキストを表示するには適していない。

% 現在我々は多くの情報の中で生活している (????)。
% SNSやニュースなどはインターネット上に流れていて簡単に見ることができるが、
% 誰が今何を考えているのか、どこにいて何をしているのかといった情報も知りたいことがある。
% それを実現するためにはありとあらゆる情報を簡単にアウトプットできて、それらを見ることが
% できるシステムが必要である。

% 現在ある、情報を見るためのインタフェースの工夫として、
% \begin{itemize}
% \item タイムライン表示
% \item インフォメーションダッシュボード
% \item スタンプ
% \end{itemize}
% の3つを紹介する。

\vspace{2mm}
\paragraph*{タイムライン表示}

Twitter, Facebook, LINEのような
近年のコミュニケーションシステムでは、
投稿を時系列に並べて表示する
「タイムライン表示」(図\ref{twitter})が広く利用されている。
%
タイムライン表示はリアルタイムに更新され、
時間順に情報を見るのには便利であるが、
投稿の多いユーザの記事ばかりが目立ちがちだし、
古い投稿がすぐに見えなくなってしまうという問題がある。

\begin{figure}[H]
\centering\fbox{\includegraphics[width=7cm]{images/twitter.png}}
\caption{Twitterのタイムライン}
\label{twitter}
\end{figure}

\vspace{1mm}
\paragraph*{スタンプ}

リアルタイムに流れていくタイムライン表示の中で情報を目立たせたいとき、
近年は「スタンプ」と呼ばれるピクトグラムが利用されることが多くなってきた(図\ref{linestamp})。

スタンプはテキストで記述するのが難しい表現や感情を伝えたり、
テキストを考えて入力するよりも速くて簡単であったりすることから、
近年LINEやFacebookメッセンジャー、オンラインゲームなどで広く利用されている。

\begin{figure}[H]
\centering\fbox{\includegraphics[width=4cm]{images/linestamp.png}}
\caption{LINEのスタンプの例}
\label{linestamp}
\end{figure}

\vspace{3mm}
そこで、このスタンプをダッシュボードのセルに表示することで、
ダッシュボードとスタンプベースのコミュニケーションを組み合わせた視覚化システム
『わかるらんど』を開発した。
『わかるらんど』は単純なアーキテクチャで、人の感情や現在の状況、
IoT機器の情報などをリアルタイムに視覚化するシステムで、
非常に汎用なダッシュボードとして利用することができる。





\input{section2}
\section{実装}
本章では『わかるらんど』の実装について述べる。

\subsection{クライアント}
クライアントはHTML/CSS/JavaScriptで実装しており、通常のブラウザ上で動作するWebアプリケーションとして動作する.

\subsection{サーバ}
サーバは並列計算プリミティブLindaをWebサーバ上に実装したlinda-server\footnote{https://github.com/node-linda/linda}を用いて実装している。

\subsubsection{Linda}
Lindaは、複数のプロセスで共有される空間を用いてプロセス間通信や
データ共有をサポートする分散並列処理を行うためのモデルである。
プロセスが共有する空間はタプル空間 (Tuple Space) と呼ばれ、
タプル空間内のデータ (Tuple) を使って通信やデータ共有を行う (図\ref{linda})。
このように Linda のモデルはきわめて単純であるが、
各クライアントやデバイス間で直接送信をする処理を記述する必要がなく,
柔軟で強力なプロセス間通信を容易に記述することができる。

\begin{figure}[h]
\centering
\fbox{
\includegraphics[width=7cm]{images/linda.png}
}
\caption{Linda}
\label{button}
\end{figure}

\subsubsection{linda-server}
linda-serverは、Node.js\footnote{https://nodejs.org}のWebSocketライブラリSocket.IO\footnote{http://socket.io}上に実装されたLindaシステムである。
linda-serverは、橋本翔氏\footnote{http://shokai.org}が開発したオープンソースソフトウェアである。
linda-serverは、\url{write, read, watch, take}の4つの基本操作によってプロセス間通信を行う。

write

新しいデータオブジェクト(タプル)を生成し共有空間(タプルスペース)に書き込む。
例えば、

ts.write(\{type: "sensor", name: "明るさ", value: 123\});

とすると、タプルスペースに、

{type: "sensor", name: "明るさ", value: 123}

というオブジェクトが書き込まれる。


read

指定した形式に部分一致するタプルがタプルスペースにあるかどうか調べて1つ読み出す。
たとえば、タプルスペースに、

{type: "sensor", name: "温度", value: 20}

{type: "sensor", name: "明るさ", value: 123}

{type: "sensor", name: "明るさ", value: 400}

が存在する状態で、

ts.read({type: "sensor", name: "明るさ"})

を実行すると、

{type: "sensor", name: "明るさ", value: 400}

が読み出される。

\url{read}はコールバックなので、一致するものが無い場合は一致するタプルが書き込まれるまで待つ。
\section{議論}

\subsection{『わかるらんど』の思想}

近年、学会などでタイムライン表示のテキストチャットが
利用される機会が増えている\cite{WISSのチャットの報告論文}。
学会チャットシステムを利用すると、
発表中に参加者が意見交換したり疑問を表明したりできるといった利点があるが、
以下のような問題も存在する。

\begin{itemize}
\item 多数の人間が同時に投稿すると投稿内容がすぐに見えなくなってしまう
\item 投稿の多いアクティブな人ばかりが目立ってしまい、消極的な参加者は議論に参加しにくい
\end{itemize}

一般に、会議などで特定の人だけが沢山発言するのはよくあることであるが、
誰もが気軽に意見を表明できる環境を構築できれば有意義であろう。

「On Air Forum」は、日本ソフトウェア科学会主催のWISSで利用されている
コミュニケーションシステムである。
「On Air Forum」のWISS2009の実証実験\cite{nishida2011}では、
全参加者の半分弱しかログインして1回以上発言していない.
WISS2015では、252アカウントが1回以上発言し総発言数は2,948回であったが、
発言数上位20\%の50アカウントによる発言が総発言数の78.1\%にあたる2,305回を占めていた(図\ref{wisschat})。
また、発言数が10回未満のアカウントは190アカウントで、これは全アカウントの75.3\%にあたる。

\begin{figure}[h]
\centering
\includegraphics[width=7cm]{images/wisschat.png}
\caption{WISS2016のチャットにおけるアカウント毎の発言数}
\label{wisschat}
\end{figure}

『わかるらんど』はユーザの表示領域が均等に決まっているため、
タイムライン表示のように投稿数が多い人ばかりが目立つということがない。
また、長いテキストを入力すると表示される文字が小さくなるので、
150人程度で利用すると長い文字は小さすぎて読めない。
必然的にユーザは図\ref{wakaruland150}のように短文を入力することを強いられる。
『わかるらんど』では長文の高度な発言は期待しておらず、
「なるほど」「わからん」「笑」などといった相槌のようなものを
視覚化してひと目で把握できるようになることを期待している。
学生、先生、企業に所属する人等様々なバックグラウンドの人が入り混じった状況で
「下手な発言ができない」「気の利いたことを言わなければならない」という
投稿を躊躇させる要素を限りなく減らし、
本当は議論に参加したいけど声が出ない/手を上げる勇気がない人でも
「なるほど」「わかる」などを『わかるらんど』に投稿することで「参加」することができる。
テキストで記述すると長くなってしまう内容も
画像スタンプを投稿することで分かってもらうことができると考える。
また、長いテキストを投稿するには適していないので『わかるらんど』を使って議論することは難しいが、
多くの会議やコンファレンスでは発表後に議論の時間が設けられているため議論はその時に行えばよい。

\begin{figure}[h]
\centering
\includegraphics[width=8cm]{images/wakaruland150.png}
\caption{150人で『わかるらんど』を使用したイメージ}
\label{wakaruland150}
\end{figure}

\subsection{研究室においての利用経験}
我々の研究室では、研究室内のディスプレイに研究室メンバー全員のリアクションと
部屋の明るさ、温度、最後に研究室のドアが開いた時間を表示した『わかるらんど』を
常に表示して、約6ヶ月の間利用してきた。
ミーティングの時間には積極的に『わかるらんど』を利用した。
普段発言の少ない人でも何らかの反応を表明したり、
誰かが面白いことを言ったときに「笑」というリアクションが並ぶと面白かった。
また発表者としても、聴衆が自分の発表を聞いてくれているかどうかわからないときに
『わかるらんど』にリアクションを投稿してもらうことで、聴衆がみんなPCの画面を見ていても
何らかの投稿があれば話を聞いているということがわかるようになった。
自分が研究室にいないときでもリアクションを投稿して、
研究室でも自宅でも研究室メンバーの気分や研究室のセンサ情報が見られるようになった。

研究室で利用しているチャットでチャットボットに話しかけることで、研究室の
\begin{itemize}
  \item ドアの鍵を開ける
  \item 照明を点ける/消す
  \item 明るさを知る
  \item 気温を知る
\end{itemize}
ことができるが、『わかるらんど』と組み合わせることで非常に便利に使うことができた。



\section{関連研究}
関連する研究には、
\begin{itemize}
\item 情報ダッシュボード
\item 多くの人がいる場での計算機を用いたコミュニケーション
\end{itemize}
に関するものがある

\subsection{情報ダッシュボードに関する研究}
研究としては,2つに区切られたレイアウトのダッシュボードのセルの配置を支援するもの
\cite{Hertzog:2015:BSP:2678025.2701383}や,
ある課題の解決のためにどのような情報をダッシュボードに表示するべきか
\cite{Jones:2015:ECI:2800835.2800963}などが議論されている.
ダッシュボードに人間の感情や現在の状況を表示するといった研究は今までに行われていない.

\subsection{多くの人がいる場での計算機を用いたコミュニケーションの研究}
「Lock-on-Chat\cite{nishida2006}」は複数の話題に分散した会話を促進するチャットシステムである。
先に述べた「On Air Forum」はリアルタイムコンテンツを視聴中のコミュニケーションシステムである。

また、消極性研究会(SIGSHY: Special Interest Group on Shyness and Hesitation around You)という
グループが、消極的な人と積極的な人が混在する場のデザインとICT支援について研究しており、
大勢の人が集まる場で消極的な人でも誰かと交流することを支援するようなシステム\cite{nishida2011}が
研究されていたり、消極性研究に関する書籍\cite{kurihara2016}も出版されている。

\section{まとめ}
本論文では、実世界の状態や人間の状況を情報ダッシュボードにわかりやすく表示し、
かつ誰もが簡単に気分などをスタンプのように投稿して共有できる「わかるらんど」システムを提案した。


\bibliographystyle{ipsjsort}
\bibliography{paper}

\end{document}
