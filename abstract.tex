\begin{abstract}
  全世界のセンサ情報やユーザの気分などを一覧表示したり投稿したりできるシステム「わかるらんど」を提案する。
  %
  ニュースや天気予報のようなリアルタイム情報を並べて一覧する
  「情報ダッシュボード」の利用が広まっているが、
  利用できる情報の種類は限られており、
  ユーザが情報を投稿して共有することはできない。
  %
  わかるらんどは、
  単純で強力なWeb上の情報共有システム「WebLinda」上に構築された
  汎用的な情報共有/視覚化システムであり、
  ユーザの気分を表明したり、チャット文字列を投稿したり、
  センサ情報やWeb上の情報を表示したり、
  ネット上のあらゆる情報を投稿/共有して一覧表示することができる。
  わかるらんどの利用により、
  情報ダッシュボードとSNSやチャットシステムを
  簡単に統合的に利用することができる。
  %
  本論文では、
  わかるらんどの思想及び利用経験について述べ、
  応用について幅広く考察する。
  
  % インフォメーションダッシュボードとスタンプベースのコミュニケーションを組み合わせた視覚化システム『わかるらんど』を提案する。
  % ニュースや天気などの情報はインターネットに流れていて誰でも簡単に見ることができるが、
  % 誰が今何を考えているのか、どこにいて何をしているのかといった情報も知りたい場合、
  % ありとあらゆる情報を簡単にアウトプットできて、それらを見ることができるシステムが必要である。
  % 『わかるらんど』はインフォメーションダッシュボードのセルにスタンプをタイル状に並べて表示する
  % ことで、人の感情や現在の状況、IoT機器の情報などをリアルタイムに視覚化するシステムで、
  % 非常に汎用なダッシュボードとして利用することができる。
  % 『わかるらんど』の実装には、Webサーバ上に実装したLindaシステムである「linda-server」を
  % 使用しており、HTTPが使用できれば様々な環境で利用することができる。
\end{abstract}
